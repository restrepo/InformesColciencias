
\section{Identificación del proyecto}
\begin{instrucciones}
  NOTA: aspectos generales sobre la forma de presentación del informe
  * El informe debe ser elaborado teniendo en cuenta la Norma
  Internacional de la American Psychologycal Association (APA) para la
  Realización de Documentos Escritos, Proyectos de Investigación y
  Trabajos de Grado.
  * Los informes se deben elaborar en tamaño carta.
  * Para la entrega de los informes, estos deben ser legajados y
  debidamente foliados o numerados, adjuntando oficio remisorio. 
  * Al documento impreso se debe adjuntar una copia del informe en medio
  magnético (CD en sobre marcado y relacionado como anexo en el oficio
  remisorio).  
  * Cada anexo debe estar numerado, haciendo referencia a lo anotado
  en el cuadro de resultados.  
  * Las publicaciones y demás productos deben presentar los debidos
  créditos a Colciencias según lo estipulado en el contrato
  respectivo.
\end{instrucciones}

\begin{longtable}{|l|l|l|}\hline
\multicolumn{3}{|l|}{1.1 Información General} \\\hline
Programa $\square$ \qquad Proyecto $\text{\rlap{$\checkmark$}}\square$ & Tipo de informe: Parcial $\square$ \qquad Final  $\text{\rlap{$\checkmark$}}\square$ & Informe No. $\text{\rlap{$3$}}\square$ de $\text{\rlap{$3$}}\square$ \\\hline
Título &\multicolumn{2}{l|}{\parbox[t]{0.58\textwidth}{} } \\ \hline
Código:  &\multicolumn{2}{l|}{} \\\hline
Número de la convocatoria  &\multicolumn{2}{l|}{} \\\hline
Número de contrato:  &\multicolumn{2}{l|}{} \\\hline
\parbox[t]{0.42\textwidth}{Programa Nacional o área de Colciencias al\par cual se encuentra adscrito el proyecto}  &\multicolumn{2}{l|}{Ciencias básicas} \\\hline
Investigador principal:   &\multicolumn{2}{l|}{Diego Alejandro Restrepo Quintero} \\\hline
Entidades ejecutoras y beneficiarias  &\multicolumn{2}{l|}{Universidad de Antioquia} \\\hline
Fecha de inicio del programa/proyecto  &\multicolumn{2}{l|}{} \\\hline
Fecha de entrega del informe   &\multicolumn{2}{l|}{\today} \\\hline
Ciudad/País  &\multicolumn{2}{l|}{Medellín/Colombia} \\\hline
\end{longtable}


%\addtocontents{toc}{\protect\enlargethispage{\baselineskip}}



%%% Local Variables: 
%%% mode: latex
%%% TeX-master: "informefinal"
%%% End: 
