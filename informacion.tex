
%CD=15
%CP=40
%PR=25
%IE=5
%P=15

\newpage
\section{Resumen }
\begin{instrucciones}
Incluya un resumen de los resultados de conocimiento obtenidos y de las principales conclusiones en máximo dos (2) páginas.
\end{instrucciones}




\section{Sinopsis técnica}
\begin{instrucciones}
Incluya una sinopsis de resultados tipo “abstract” con fines divulgativos en un máximo de 500 palabras. Si usted lo considera conveniente, envíe además la
sinopsis en inglés. Informe a Colciencias si autoriza su publicación, en caso de ser necesario. 
\end{instrucciones}

{}\footnote{Se autoriza la publicación de la Sinopsis técnica.}

\pagestyle{plain}
\begin{landscape}
  \section{Cumplimiento de objetivos}

  \subsection{Cumplimiento de objetivos generales}
%   \begin{instrucciones}
%     Para cada uno de los objetivos generales del programa/proyecto (en caso de que exista más de uno) establezca su grado de cumplimiento y proporcione una
% explicación sobre el mismo. De no haberse cumplido el objetivo o los objetivos generales, elabore una justificación técnica por la que esta situación se presentó y
% explique si en lugar del(os) objetivo(s) planteado(s) en un principio se obtuvo algún otro resultado. Incluya para cada objetivo general una tabla según las
% instrucciones dadas a continuación
%   \end{instrucciones}

\renewcommand{\arraystretch}{1.5}

%the width must add 1.2
\gdef\ic{0.2}  \gdef\iic{0.3} \gdef\iiic{0.3} 
\FPeval{\ivc}{1.2-\ic-\iic-\iiic}
\FPeval{\iict}{\iic+\iiic}
\hspace{-1.4cm}\begin{longtable}{|p{\ic\textwidth}|p{\iic\textwidth}|p{\iiic\textwidth}|p{\ivc\textwidth}|}
\hline
\gdef\objetivo{OBJETIVO GENERAL}
\gdef\objetivotxt{
}
\gdef\porcentaje{100}
\gdef\resultado{RESULTADO OBTENIDO}
\gdef\anexo{ANEXO SOPORTE DEL DESARROLLO Y OBTENCIÓN DE RESULTADOS}
\gdef\dificultades{DIFICULTADES}
\gdef\observaciones{OBSERVACIONES}

\objetivo & \multicolumn{2}{p{\iict\textwidth}|}{\objetivotxt} & 
\parbox{\ivc\textwidth}{
\begin{tabular}{l|r}
 \% de cumplimiento & \porcentaje\% \\
\end{tabular}
}\\ \hline
\resultado &\anexo  &\dificultades  & \observaciones  \\\hline
\gdef\resultado{}
\gdef\anexo{}
\gdef\dificultades{}
\gdef\observaciones{}
\resultado &\anexo  &\dificultades  & \observaciones  \\\hline
\end{longtable}


  \subsection{Cumplimiento de objetivos específicos}
%   \begin{instrucciones}
%     Para cada uno de los objetivos específicos del programa/proyecto establezca su grado de cumplimiento y proporcione una explicación sobre el mismo. En caso
% de no haberse cumplido el objetivo o los objetivos específicos, elabore una justificación técnica por la que esta situación se presentó y si en lugar del(os)
% objetivo(s) planteado(s) en un principio se obtuvo algún otro resultado. Incluya para cada objetivo específico una tabla según las instrucciones dadas a
% continuación:
%   \end{instrucciones}
%the width must add 1.2
\newcommand{\ObjetivoEspecifico}[6][100]{
%
\gdef\objetivo{OBJETIVO ESPECÍFICO:}
\gdef\objetivotxt{#2}
\gdef\porcentaje{#1}
\gdef\resultado{\centering RESULTADO OBTENIDO}
\gdef\producto{\centering PRODUCTO (si aplica)}
\gdef\anexo{\centering ANEXO SOPORTE DEL  DESARROLLO Y OBTENCIÓN DE RESULTADOS}
\gdef\observaciones{ OBSERVACIONES }

\objetivo & \multicolumn{2}{p{\iict\textwidth}|}{\objetivotxt} & 
\parbox{\ivc\textwidth}{
\begin{tabular}{l|r}
 \% de cumplimiento & \porcentaje \% \\
\end{tabular}
}\\ \hline
\resultado & \producto  &\anexo  & \observaciones  \\\hline
%
\gdef\resultado{#3}
\gdef\producto{
\parbox[t]{\iic\textwidth}{
#4
 }
}
\gdef\anexo{
\parbox[t]{\iiic\textwidth}{
#5
}
}
\gdef\observaciones{#6}
\resultado & \producto  &\anexo  & \observaciones  \\\hline
}


\gdef\ic{0.3}  \gdef\iic{0.4} \gdef\iiic{0.2} 
\FPeval{\ivc}{round(1.2-\ic-\iic-\iiic,2)}
\FPeval{\iict}{round(\iic+\iiic,2)}

\hspace{-1.4cm}\begin{longtable}{|p{\ic\textwidth}|p{\iic\textwidth}|p{\iiic\textwidth}|p{\ivc\textwidth}|}\hline
%
\ObjetivoEspecifico%
%[porcentaje]
{%objetivo: 
}
{%\resultado 
}
{%\producto  
}
{%\anexo  
}
{%\observaciones 
}
%
% %
% \ObjetivoEspecifico%
% %[porcentaje]
% {%objetivo:
% }
% {%\resultado 
% }
% {%\producto  
% }
% {%\anexo  
% }
% {%\observaciones 
% }
% %

\end{longtable}




\end{landscape}
\begin{landscape}
  \subsection{Descripción de otros resultados obtenidos}
  % Proporcione una descripción del grado de cumplimiento de los otros
  % resultados esperados que se hayan pactado contractualmente. En caso
  % de haber expresado estos resultados en términos de indicadores, se
  % debe proporcionar una comparación del estado inicial y final de los
  % mismos. Si en el desarrollo del proyecto/programa surgen resultados
  % adicionales, relaciónelos y haga la aclaración correspondiente. Para
  % cada resultado incluya una fila según las instrucciones dadas a
  % continuación:

%\begin{tabular}{|p{9cm}|l|l|l|}\hline
\hspace{-1cm}  \begin{longtable}{|p{0.23\textwidth}|p{0.25\textwidth}|p{0.65\textwidth}|p{0.1\textwidth}|}\hline
 \parbox[t]{0.23\textwidth}{\centering OTROS RESULTADOS\\ (comprometidos\\ contractualmente)} &
\centering  INDICADOR DE \par
CUMPLIMIENTO &\centering DESCRIPCIÓN DEL \par
RESULTADO OBTENIDO   &\centering ANEXO\par SOPORTE\endhead\hline
 & &  &\\\hline
%Otros (especificar)  & & &\\\hline
  \end{longtable}
  
\section{Resultados Adicionales }
No se reportan resultados adicionales.


\section{Cumplimiento de la metodología}
La metodología se siguió sin modificaciones. 

\section{Cronograma de ejecución}


\gdef\ic{0.24}  \gdef\iic{0.24} \gdef\iiic{0.24}  \gdef\ivc{0.24} 
\FPeval{\vc}{round(1.2-\ic-\iic-\iiic-\ivc,2)}

\newcommand{\Cronograma}[5]{
\gdef\actividad{#1}
\gdef\objetivo{#2}
\gdef\fecha{#3}
\gdef\cambios{#4}
\gdef\plan{#5}
\actividad & \objetivo &\fecha  & \cambios & \plan  \\ \hline
}

\hspace{-1.4cm}\begin{longtable}{|p{\ic\textwidth}|p{\iic\textwidth}|p{\iiic\textwidth}|p{\ivc\textwidth}|p{\vc\textwidth}|}
 \hline
\Cronograma%
{%actividad
\centering ACTIVIDADES
}%
{%objetivo
\centering OBJETIVO\par RELACIONADO
}%
{%fecha
\centering FECHA DE\par EJECUCIÓN
}%
{%cambios
\centering CAMBIOS \par SOLICITADOS Y \par APROBADOS POR \par COLCIENCIAS \par (si aplica)
}%
{%plan
PLAN DE CONTINGENCIA \par 
(si aplica)
}
%
\Cronograma%
{%actividad
}%
{%objetivo
}%
{%fecha
}%
{%cambios
}%
{%plan
}
%
\end{longtable}



\end{landscape}

\pagestyle{fancy}
\section{Proyección de los resultados obtenidos frente a los impactos registrados en el proyecto/programa (si
aplica)}
No aplica.
\section{Aspectos financieros  }
Ver anexos correspondientes.
\section{Discusión y análisis  }
\begin{instrucciones}
  Realice el análisis y discusión de los resultados obtenidos en la ejecución del programa/proyecto, en una extensión máxima de 3 a 5 páginas.
\end{instrucciones}

\section{Conclusiones          }
\begin{instrucciones}
  Enuncie las conclusiones de acuerdo a los resultados obtenidos. Incluya las posibles perspectivas de investigación o continuidad del programa/proyecto.
\end{instrucciones}

\section{Siglas y abreviaturas }
\begin{itemize}
\item[LHC:] Siglas en inglés del Gran Colisionador de Hadrónes
\end{itemize}

%and eprint 1303.0278 and eprint 1302.5298 and eprint 1212.3310

%generacion de conocimiento punta a nivel mundial




%%% Local Variables: 
%%% mode: latex
%%% TeX-master: "informefinal"
%%% ispell-local-dictionary: "castellano8"
%%% End: 
